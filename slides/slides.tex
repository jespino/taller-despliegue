\documentclass[10pt]{beamer}

\usepackage[latin1]{inputenc}
\usepackage[spanish]{babel}

\mode<presentation>
\usetheme{Madrid}
\usecolortheme[RGB={128,0,0}]{structure}
\usefonttheme{serif}
\useinnertheme{rectangles}
\useoutertheme{split}
\setbeamercovered{transparent}

\title{Taller: Desplegando Django}
\author{Jes�s Espino Garc�a}
\date{21 de Julio de 2014}
\subject{Taller: Desplegando Django}

\institute[Kaleidos]{\includegraphics[height=1.5cm]{kaleidos}}

\begin{document}

  \frame{\maketitle}

  \section{Parte I: Despliegue en local}

  \begin{frame}{�Que vamos a hacer?}
    \begin{itemize}
      \item Montaremos un entorno local en nuestra maquina.
      \item Montaremos este entorno en una maquina virtual.
      \item Desplegaremos ese entorno en la maquina virtual.
      \item Actualizaremos ese entorno usando fabric.
    \end{itemize}
  \end{frame}

  \begin{frame}{Instalamos nuestros requisitos de sistemas}
    \begin{itemize}
      \item apt-get install python3 virtualenvwrapper git
    \end{itemize}
  \end{frame}

  \begin{frame}{Clonamos el repositorio}
    \begin{itemize}
        \item Usamos git para ayudar a desplegar el c�digo
        \item Nos permite ver si a habido cambios en el c�digo f�cilmente.
    \end{itemize}
  \end{frame}

  \begin{frame}{Clonamos el repositorio}
    \begin{itemize}
      \item Clonamos el repositorio de github jespino/taller-despliegue
    \end{itemize}
  \end{frame}

  \begin{frame}{Creamos un virtualenv}
    \begin{itemize}
      \item Usamos virtualenv.
      \item Nos aisla de las bilbiotecas del sistema.
      \item Nos permite tener entornos muy dispares.
    \end{itemize}
  \end{frame}

  \begin{frame}{Creamos un virtualenv}
    \begin{itemize}
      \item Instalamos virtualenvwrapper.
      \item Creamos un virtualenv usando python 3.
    \end{itemize}
  \end{frame}

  \begin{frame}{Instalamos los requisitos}
    \begin{itemize}
      \item Es muy comun usar un fichero requirements.txt
      \item En este definimos nuestros requisitos.
    \end{itemize}
  \end{frame}

  \begin{frame}{Instalamos los requisitos}
    \begin{itemize}
      \item pip install -r requirements.txt
    \end{itemize}
  \end{frame}

  \begin{frame}{Eejecutamos nuestra aplicaci�n}
    \begin{itemize}
      \item Mientras desarrollamos usamos el servidor integrado de django.
      \item Para producci�n solemos usar gunicorn.
    \end{itemize}
  \end{frame}

  \begin{frame}{Ejecutamos nuestra aplicaci�n}
    \begin{itemize}
      \item Ejecutamos la aplicaci�n con el servidor integrado de django.
      \item Ejecutamos la aplicaci�n con gunicorn
    \end{itemize}
  \end{frame}

  \begin{frame}{�Que no hemos hecho?}
    \begin{itemize}
      \item Crear la bbdd (syncdb o migrate).
      \item Cargar fixtures.
    \end{itemize}
  \end{frame}

  \section{Parte II: Despliegue en remoto}

  \begin{frame}{Arrancar la maquina virtual}
    \begin{itemize}
      \item En Kaleidos solemos desplegar en ubuntu server.
      \item Copiamos la maquina virtual.
      \item Le configuramos la red y la memoria.
      \item Instalamos OpenSSH en la maquina virtual.
    \end{itemize}
  \end{frame}

  \begin{frame}{Preparamos nuestro servidor}
    \begin{itemize}
      \item Creamos un usario para desplegar el proyecto.
      \item Montamos el entorno como en la Parte I.
    \end{itemize}
  \end{frame}

  \begin{frame}{Arrancar nuestro servicio}
    \begin{itemize}
      \item Configuramos supervisor para que arranque gunicorn por nosotros.
    \end{itemize}
  \end{frame}

  \begin{frame}{Lo colocamos detras de un nginx}
    \begin{itemize}
      \item Configuramos el nginx para que haga de proxy
    \end{itemize}
  \end{frame}

  \begin{frame}{Servimos los estaticos y medias con nginx}
    \begin{itemize}
      \item Configuramos el nginx para que sirva estaticos y medias.
    \end{itemize}
  \end{frame}

  \begin{frame}{Usamos socket unix entre nginx y gunicorn}
    \begin{itemize}
      \item Configuramos el gunicorn para que sirva en un socket unix.
      \item Configuramos el nginx para que haga proxy del socket unix.
    \end{itemize}
  \end{frame}

  \begin{frame}{Configuramos nuestro servicio de correo}
    \begin{itemize}
      \item Instalamos y configuramos postfix.
    \end{itemize}
  \end{frame}

  \begin{frame}{Configuramos celery}
    \begin{itemize}
      \item Instalamos RabbitMQ.
      \item Configuramos el celery en el supervisor.
    \end{itemize}
  \end{frame}

  \section{Parte III: Despliegue automatico}

  \begin{frame}{Desplegamos con fabric}
    \begin{itemize}
      \item Actualizamos automaticamente para evitar errores.
      \item Y por ahorrar tiempo.
    \end{itemize}
  \end{frame}

  \begin{frame}{Desplegamos con fabric}
    \begin{itemize}
      \item Creamos un script de fabric para actualizar.
    \end{itemize}
  \end{frame}

  \begin{frame}
    \frametitle{\begin{center}Fin\end{center}}
    \begin{center}
      \includegraphics[height=3cm]{kaleidos}
    \end{center}
  \end{frame}

\end{document}
